% -*- latex -*-
%%%%%%%%%%%%%%%%%%%%%%%%%%%%%%%%%%%%%%%%%%%%%%%%%%%%%%%%%%%%%%%%
%%%%
%%%% This TeX file is part of the course
%%%% Introduction to Scientific Programming in C++/Fortran2003
%%%% copyright 2017 Victor Eijkhout eijkhout@tacc.utexas.edu
%%%%
%%%% arrayf.tex : arrays in Fortran
%%%%
%%%%%%%%%%%%%%%%%%%%%%%%%%%%%%%%%%%%%%%%%%%%%%%%%%%%%%%%%%%%%%%%

Array handling in Fortran is similar to C++ in some ways, but there
are differences, such as that  Fortran indexing starts at~1, rather
than~0. More importantly, Fortran has better handling of
multi-dimensional arrays.

\Level 0 {Static arrays}

The preferred way for specifying an array size is:

\begin{block}{Fortran dimension}
  \label{sl:farray-dimension}
\begin{verbatim}
real(8), dimension(100) :: x,y
integer :: i(10,20)
\end{verbatim}
  Static, obey scope.
\end{block}

Such an array, with size explicitly indicated, is called a
\indextermsub{static}{array} or \indextermsub{automatic}{array}.
(See section~\ref{sec:allocatable} for
dynamic arrays.)

Array indexing in Fortran is 1-based:
\begin{block}{1-based Indexing}
  \label{sl:farray-base1}
\begin{verbatim}
integer,parameter :: N=8
real(4),dimension(N) :: x
do i=1,N
  ... x(i) ...
\end{verbatim}
\end{block}

Unlike C++, Fortran can specify the lower bound explicitly:
\begin{block}{Lower bound}
  \label{sl:farray-lower}
\begin{verbatim}
real,dimension(-1:7) :: x
do i=-1,7
  ... x(i) ...
\end{verbatim}
\end{block}

Such arrays, as in~C++, obey the scope: they disappear at the end of
the program or subprogram.

\Level 1 {Initialization}

There are various syntaxes for \indextermbus{array}{initialization},
including the use of implicit do-loops:
\begin{block}{Array initialization}
  \label{sl:farray-init}
  \verbatimsnippet{farrayinit}
\end{block}

\Level 1 {Sections}

\begin{block}{Array sections}
  \label{sl:farray-section}
  Use the colon notation to indicate ranges:
\begin{verbatim}
real(4),dimension(5) :: x
x(2:5) = x(1:4)
\end{verbatim}
\end{block}

\begin{block}{Use of sections}
  \label{sl:farray-sectionassign}
  \snippetwithoutput{fsectionassign}{arrayf}{sectionassign}
\end{block}

You can even use a stride:

\begin{block}{Strided sections}
  \label{sl:farray-strideassign}
  \snippetwithoutput{fsectionmg}{arrayf}{sectionmg}
\end{block}

\Level 1 {Integer arrays as indices}

\begin{block}{Index arrays}
  \label{sl:farray-indexarray}
\begin{verbatim}
integer,dimension(4) :: i = [2,4,6,8]
real(4),dimension(10) :: x
print *,x(i)
\end{verbatim}
\end{block}

\Level 0 {Multi-dimensional}

Arrays above had `\emph{rank}\index{array!rank}~one'. The rank is
defined as the number of indices you need to address the
elements. Calling this the `dimension' of the array can be confusing, but
we will talk about the first and second dimension of the array.

A rank-two array, or matrix, is defined like this:
\begin{block}{Multi-dimension arrays}
  \label{sl:farray-2d}
\begin{verbatim}
real(8),dimension(20,30) :: array
array(i,j) = 5./2
\end{verbatim}
\end{block}

\begin{block}{Array layout}
  \label{sl:farray-layout}
  Sometimes you have to take into account how a higher rank array
  is laid out in (linear) memory:

  \includegraphics[scale=.1]{arraybycol}

  `First index varies quickest'
\end{block}

\begin{block}{Array printing}
  \label{sl:farray-print}
  Fill array by rows:
  \[ \begin{matrix}1&2&\ldots&N\\ N+1&\ldots\\ &\ldots\\ &&&MN
  \end{matrix}
  \]
  %
  \snippetwithoutput{printarray}{arrayf}{printarray}
\end{block}

\Level 1 {Querying an array}

We have the following properties of an array:
\begin{itemize}
\item The bounds are the lower and upper bound in each dimension.
  For instance, after
\begin{verbatim}
integer,dimension(-1:1,-2:2) :: symm
\end{verbatim}
the array \n{symm} has a lower bound of~\n{-1} in the first dimension
and \n{-2} in the second. The functions \indextermttdef{Lbound} and
\indextermttdef{Ubound} give these bounds as array or scalar:
\begin{verbatim}
array_of_lower = Lbound(symm)
upper_in_dim_2 = Ubound(symm,2)
\end{verbatim}

\snippetwithoutput{fsection2}{arrayf}{fsection2}

\item The \emph{extent}\index{extent!of array dimension} is the number
  of elements in a certain dimension, and the
  \emph{shape}\index{shape!of array} is the array of extents.

\item The \indextermttdef{size} is the number of elements, either for
  the whole array, or for a specified dimension.
\begin{verbatim}
integer :: x(8), y(5,4)
size(x)
size(y,2)
\end{verbatim}
\end{itemize}

\begin{slide}{Query functions}
  \label{sl:farray-query}
    \begin{itemize}
    \item Bounds: \n{lbound}, \n{ubound}
    \item \n{size}
\begin{verbatim}
integer :: x(8), y(5,4)
size(x)
size(y,2)
\end{verbatim}
    \end{itemize}
\end{slide}

\Level 1 {Reshaping}

\indextermfort{RESHAPE}
\begin{verbatim}
  array = RESHAPE( list,shape )
\end{verbatim}
Example:
\verbatimsnippet{shape44}

\indextermfort{SPREAD}
\begin{verbatim}
  array = SPREAD( old,dim,copies )
\end{verbatim}

\Level 0 {Arrays to subroutines}

Subprogram needs to know the shape of an array, not the actual bounds:

\begin{block}{Pass array to subroutine}
  \label{sl:farray-pass1d}
  \verbatimsnippet{fpass1d}
\end{block}

The array inside the subroutine is known as a
\indextermsub{assumed-shape}{array} or
\indextermsub{automatic}{array}.

\Level 0 {Allocatable arrays}
\label{sec:allocatable}

Static arrays are fine at small sizes. However, 
there are two main arguments against using them at large sizes.
\begin{itemize}
\item Since the size is explicitly stated, it makes your program
  inflexible, requiring recompilation to run it with a different
  problem size.
\item Since they are allocated on the so-called \indexterm{stack},
  making them too large can lead to \indextermbus{stack}{overflow}.
\end{itemize}

A better strategy is to indicate the shape of the array, and use
\indextermtt{allocate} to specify
the size later, presumably in terms of run-time program parameters.

\begin{block}{Array allocation}
  \label{sl:farray-alloc}
\begin{verbatim}
real(8), dimension(:), allocatable :: x,y

n = 100
allocate(x(n), y(n))
\end{verbatim}
You can \indextermtt{deallocate} the array when you don't need the
space anymore.
\end{block}

If you are in danger of running out of memory, it can be a good idea
to add a \n{stat=ierror} clause to the \n{allocate} statement:
\begin{verbatim}
integer :: ierr
allocate( x(n), stat=ierr )
if ( ierr/=0 ) ! report error
\end{verbatim}

Has an array been allocated:
\begin{verbatim}
Allocated( x ) ! returns logical
\end{verbatim}

Allocatable arrays are automatically deallocated when they go out of
scope. This prevents the \indextermbus{memory}{leak} problems of~C++.

Explicit deallocate:
\begin{verbatim}
deallocate(x)
\end{verbatim}

\Level 0 {Array slicing}

Fortran is more sophisticated than C++ in how it can handle arrays as
a whole. For starters, you can assign one array to another:
\begin{verbatim}
real(4),dimension(25,26) :: a,b
a = b
\end{verbatim}
You can assign subarrays, or \emph{array slices}\index{index!slice},
as long as they have the same shape. You use colon-syntax to indicate
ranges:

\begin{block}{Array slicing}
  \label{sl:farray-slices}
  \begin{itemize}
  \item {\tt :} %\n{:}
    to get all indices,
  \item {\tt :n} % \n{:n}
    to get indices up to~\n{n},
  \item {\tt n:} % \n{n:}
    to get indices \n{n} and up.
  \item {\tt m:n}
    indices in range \n{m,...,n}.
  \end{itemize}
\end{block}

For multi-dimensional arrays, you need to indicate a range in all
dimensions.

\begin{block}{Array slicing in multi-D}
  \label{sl:farray-sliced}
\begin{verbatim}
real(8),dimension(10) :: a,b
a(1:9) = b(2:10)
\end{verbatim}
or
\begin{verbatim}
logical,dimension(25,3) :: a
logical,dimension(25)   :: b
a(:,2) = b
\end{verbatim}
You can also use strides.
\end{block}

\begin{exercise}
  \label{ex:farray-shift}

  \includegraphics[scale=.3]{yx_average}

  Code $\forall_i\colon y_i=(x_i+x_{i+1})/2$:
  \begin{itemize}
  \item First with a do loop; then
  \item in a single array assignment statement by using sections.
  \end{itemize}
  Initialize the array \n{x} with values that allow you to check the
  correctness of your code.
\end{exercise}

\Level 0 {Array output}

Use implicit do-loops; section~\ref{sec:f-impdo}.

\Level 0 {Operating on an array}

\Level 1 {Arithmetic operations}

Between arrays of the same shape:
\begin{verbatim}
A = B+C
D = D*E
\end{verbatim}
(where the multiplication is by element).

\Level 1 {Intrinsic functions}

The following intrinsic functions are avaible for arrays:
\begin{block}{Array intrinsics}
  \label{sl:array-funcf}
  \begin{itemize}
  \item \indextermtt{MaxVal} finds the maximum value in an array.
  \item \indextermtt{MinVal} finds the minimum value in an array.
  \item \indextermtt{Sum} returns the sum of all elements.
  \item \indextermtt{Product} return the product of all elements.
  \item \indextermfort{MaxLoc} returns the index of the maximum
    element.
\begin{verbatim}
i = MAXLOC( array [, mask ] )
\end{verbatim}
  \item \indextermtt{MinLoc} returns the index of the minimum element.
  \item \indextermtt{MatMul} returns the matrix product of two matrices.
  \item \indextermtt{Dot_Product} returns the dot product of two
    arrays.
  \item \indextermtt{Transpose} returns the transpose of a matrix.
  \item \indextermtt{Cshift} rotates elements through an array.
  \end{itemize}
\end{block}

\begin{exercise}
  \label{ex:fmatnorm}
  The 1-norm of a matrix is defined as the maximum sum of absolute
  values in any column:
  \[ \|A\|_1 = \max_j \sum_i |A_{ij}| \]
  while the infinity-norm is defined as the maximum row sum:
  \[ \|A\|_\infty = \max_i \sum_j |A_{ij}| \]
  Implement these functions using array intrinsics.
\end{exercise}

\begin{exercise}
  \label{ex:fmatmul}
  Compare implementations of the matrix-matrix product.
  \begin{enumerate}
  \item Write the regular \n{i,j,k} implementation, and store it as
    reference.
  \item Use the \n{DOT_PRODUCT} function, which eliminates the \n{k}
    index. How does the timing change? Print the maximum absolute
    distance between this and the reference result.
  \item Use the \n{MATMUL} function. Same questions.
  \item Bonus question: investigate the \n{j,k,i} and \n{i,k,j}
    variants. Write them both with array sections and individual array
    elements. Is there a difference in timing?
  \end{enumerate}
  Does the optimization level make a difference in timing?
\end{exercise}

\Level 1 {Restricting with \tt{where}}

\indextermfort{where}
\begin{verbatim}
where ( A<0 ) B = 0
\end{verbatim}

Full form:
\begin{verbatim}
WHERE ( logical argument )
  sequence of array statements
ELSEWHERE
  sequence of array statements
END WHERE
\end{verbatim}

\Level 1 {Global condition tests}

Reduction of a test on all array elements:
\indextermfort{all}
\begin{verbatim}
REAL(8),dimension(N,N) :: A
LOGICAL :: positive,positive_row(N),positive_col(N)
positive = ALL( A>0)
positive_row = ALL( A>0,1 )
positive_col = ALL( A>0,2 )
\end{verbatim}

\begin{exercise}
  Use array statements (that is, no loops) to fill a two-dimensional
  array~\n{A} with random numbers between zero and~one. Then fill two
  arrays \n{Abig} and \n{Asmall} with the elements of~\n{A} that are
  great than~$0.5$, or less than~$0.5$ respectively:
  \[ A_{\scriptstyle\mathrm{big}}(i,j) =
  \begin{cases}
    A(i,j)&\hbox{if $A(i,j)\geq 0.5$}\\ 0&\hbox{otherwise}
  \end{cases}
  \]
  \[ A_{\scriptstyle\mathrm{small}}(i,j) =
  \begin{cases}
    0&\hbox{if $A(i,j)\geq 0.5$}\\ A(i,j)&\hbox{otherwise}
  \end{cases}
  \]
  Using more array statements, add \n{Abig} and \n{Asmall}, and test
  whether the sum is close enough to~\n{A}.
\end{exercise}

Similar to \n{all}, there is a function~\indextermfort{any} that tests
if any array element satisfies the test.
\begin{verbatim}
if ( Any( Abs(A-B)>
\end{verbatim}

\Level 0 {Array operations}

\Level 1 {Loops without looping}

In addition to ordinary do-loops, Fortran has mechanisms that save you
typing, or can be more efficient in some circumstances.
\begin{itemize}
\item Slicing: if your loop assigns to an array from another array,
  you can use slice notation:
\begin{verbatim}
a(:) = b(:)
c(1:n) = d(2:n+1)
\end{verbatim}
\item The \indextermtt{forall} keyword also indicates an array assignment:
\begin{verbatim}
forall (i=1:n)
  a(i) = b(i)
  c(i) = d(i+1)
end forall
\end{verbatim}
You can tell that this is for arrays only, because the loop index has
to be part of the left-hand side of every assignment.

This mechanism is prone to misunderstanding and therefore now
deprecated.
It is not a parallel loop! For that, the following mechanism is preferred.
\item The \indexterm{do concurrent} is a true do-loop. With the
  \n{concurrent} keyword the user specifies that the
  iterations of a loop are independent, and can therefore possibly be
  done in parallel:
\begin{verbatim}
do concurrent (i=1:n)
  a(i) = b(i)
  c(i) = d(i+1)
end do
\end{verbatim}
\end{itemize}

\Level 1 {Loops without dependencies}

Here are some illustrations of simple array copying with the above
mechanisms.

\verbatimsnippet{blockrecur}
\begin{verbatim}
Original     1   2   3   4   5   6   7   8   9  10
Recursive    0   2   4   6   8  10  12  14  16  18
\end{verbatim}

\verbatimsnippet{blockslice}
\begin{verbatim}
Original     1   2   3   4   5   6   7   8   9  10
Sliced       0   2   4   6   8  10  12  14  16  18
\end{verbatim}

\verbatimsnippet{blockforall}
\begin{verbatim}
Original     1   2   3   4   5   6   7   8   9  10
Forall       0   2   4   6   8  10  12  14  16  18
\end{verbatim}

\verbatimsnippet{blockconc}
\begin{verbatim}
Original     1   2   3   4   5   6   7   8   9  10
Concurrent   0   2   4   6   8  10  12  14  16  18
\end{verbatim}

\begin{exercise}
  Create arrays \n{A,C} of length~\n{2N}, and \n{B}~of length~\n{N}.
  Now implement
  \[ B_i = (A_{2i}+A_{2i+1})/2,\quad i=1,\ldots N \]
  and
  \[ C_i = A_{i/2},\quad i=1,\ldots 2N \]
  using all four mechanisms. Make sure you get the same result every time.
\end{exercise}

\Level 1 {Loops with dependencies}

For parallel execution of a loop, all iterations have to be independent.
This is not the case if the loop has a \indexterm{recurrence}, and in
this case, the `do concurrent' mechanism is not appropriate.
%
Here are the above four constructs, but applied to a loop with a dependence.
%
\verbatimsnippet{slicerecur}
%
\begin{verbatim}
Original   1   2   3   4   5   6   7   8   9  10
Recursiv   1   2   4   8  16  32  64 128 256 512
\end{verbatim}

The slicing version of this:
%
\verbatimsnippet{sliceslice}
%
\begin{verbatim}
Original   1   2   3   4   5   6   7   8   9  10
Sliced     1   2   4   6   8  10  12  14  16  18
\end{verbatim}
%
acts as if the right-hand side is saved in a temporary array, and
subsequently assigned to the left-hand side.

Using `forall' is equivalent to slicing:
%
\verbatimsnippet{sliceforall}
%
\begin{verbatim}
Original     1   2   3   4   5   6   7   8   9  10
Forall       1   2   4   6   8  10  12  14  16  18
\end{verbatim}

On the other hand, `do concurrent' does not use temporaries, so it is
more like the sequential version:
%
\verbatimsnippet{sliceconc}
%
\begin{verbatim}
Original     1   2   3   4   5   6   7   8   9  10
Concurrent   1   2   4   8  16  32  64 128 256 512
\end{verbatim}
Note that the result does not have to be equal to the sequential
execution: the compiler is free to rearrange the iterations any way it
sees fit.

\Level 0 {Review questions}

\begin{exercise}
  \label{ex:farray-assign-slice}
  Let the following declarations be given, and assume that all arrays
  are properly initialized:
\begin{verbatim}
real                   :: x
real, dimension(10)    :: a, b
real, dimension(10,10) :: c, d
\end{verbatim}

Comment on the following lines: are they legal, if so what do they do?
\begin{enumerate}
\item \n{a = b}
\item \n{a = x}
\item \n{a(1:10) = c(1:10)}
\end{enumerate}

How would you:
\begin{enumerate}
\item Set the second row of \n{c} to~\n{b}?
\item Set the second row of \n{c} to the elements of~\n{b}, last-to-first?
\end{enumerate}
\end{exercise}
