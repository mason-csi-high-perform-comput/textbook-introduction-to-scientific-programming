% -*- latex -*-
%%%%%%%%%%%%%%%%%%%%%%%%%%%%%%%%%%%%%%%%%%%%%%%%%%%%%%%%%%%%%%%%
%%%%
%%%% This TeX file is part of the course
%%%% Introduction to Scientific Programming in C++/Fortran2003
%%%% copyright 2017 Victor Eijkhout eijkhout@tacc.utexas.edu
%%%%
%%%% efficiency.tex : ruminations on efficiency
%%%%
%%%%%%%%%%%%%%%%%%%%%%%%%%%%%%%%%%%%%%%%%%%%%%%%%%%%%%%%%%%%%%%%

\Level 0 {Order of complexity}

\Level 1 {Time complexity}
\label{sec:time_complex}

\begin{exercise}
  For each number~$n$ from 1 to~100, print the sum of all numbers 1~through~$n$.
\end{exercise}

There are several possible solutions to this exercise. Let's assume
you don't know the formula for the sum of the numbers~$1\ldots n$.
You can have a solution that keeps a running sum, and a solution with
an inner loop.

\begin{exercise}
  How many operations, as a function of~$n$, are performed in these
  two solutions?
\end{exercise}

\Level 1 {Space complexity}

\begin{exercise}
  Read numbers that the user inputs; when the user inputs zero or
  negative, stop reading. Add up all the positive numbers
  and print their average. 
\end{exercise}

This exercise can be solved by storing the numbers in a
\n{std::vector}, but one can also keep a running sum and count.

\begin{exercise}
  How much space do the two solutions require?
\end{exercise}
