% -*- latex -*-
%%%%%%%%%%%%%%%%%%%%%%%%%%%%%%%%%%%%%%%%%%%%%%%%%%%%%%%%%%%%%%%%
%%%%
%%%% This TeX file is part of the course
%%%% Introduction to Scientific Programming in C++/Fortran2003
%%%% copyright 2017 Victor Eijkhout eijkhout@tacc.utexas.edu
%%%%
%%%% elementsf.tex : basic language elements of Fortran
%%%%
%%%%%%%%%%%%%%%%%%%%%%%%%%%%%%%%%%%%%%%%%%%%%%%%%%%%%%%%%%%%%%%%

Fortran is an old programming language, dating back to the 1950s, and
the first `high level programming language' that was widely used.
In a way, the fields of programming language design and compiler
writing started with Fortran, rather than this language being based on
established fields. Thus, the design of Fortran has some
idiosyncracies that later designed languages have not adopted. Many of
these are now `deprecated' or simply inadvisable. Fortunately, it is
possible to write Fortran in a way that is every bit as modern and
sophisticated as other current languages.

In this part of our book, we will teach you safe practices for
writing Fortran. Occasionally we will not mention practices that you
will come across in old Fortran codes, but that we would not advise
you taking up. While our exposition of Fortran can stand on its own,
we will in places point out explicitly differences with~C++.

For secondary reading, this is a good course on modern Fortran:
\url{http://www.pcc.qub.ac.uk/tec/courses/f77tof90/stu-notes/f90studentMIF_1.html}

\Level 0 {Compiling Fortran}

For Fortran programs, the compiler is \indextermtt{gfortran} for the
GNU compiler, and \indextermtt{ifort} for Intel.
Fortran programs can have a number of extensions, but some of them
have special meaning to the compiler, by convention. In this book we
adopt the \indextermtt{F90} extension.

The minimal Fortran program is:
%
\verbatimsnippet{emptyf}

\begin{exercise}
  Add the line
\begin{verbatim}
print *,"Hello world!"
\end{verbatim}
to the empty program, and compile and run it.
\end{exercise}

\Level 0 {Main program}

Fortran does not use curly brackets to delineate blocks, instead you
will find \indextermtt{end} statements. The very first one appears
right when you start writing your program:
a~Fortran program needs to start with a \n{Program} line, and end with
\n{End Program}. The program needs to have a name on both lines:
%
\verbatimsnippet{emptyf}
%
and you can not use that name for any entities in the program.

\Level 1 {Program structure}

Unlike C++, Fortran can not mix variable declarations and executable
statements, so both the main program and any subprograms have roughly
a
structure:
\begin{verbatim}
Program foo
  < declarations >
  < statements >
End Program foo
\end{verbatim}
(The \indexterm{emacs} editor will supply the block type and name if
you supply the `end' and hit the TAB or RETURN key; see
section~\ref{sec:editor-mode}.)

\begin{slide}{Program structure}
  \label{sl:programf}
\begin{verbatim}
Program foo
  < declarations >
  < statements >
End Program foo
\end{verbatim}
\end{slide}

\Level 1 {Statements}

Let's say a word about layout. Fortran has a `one line, one statement'
principle.
\begin{itemize}
\item As long as a statement fits on one line, you don't have to
  terminate it explicitly with something like a semicolon:
\begin{verbatim}
x = 1
y = 2
\end{verbatim}
\item If you want to put two statements on one line, you have to
  terminate the first one:
\begin{verbatim}
x = 1; y = 2
\end{verbatim}
\item If a statement spans more than one line, all but the first line
  need to have an explicit \indexterm{continuation character}, the ampersand:
\begin{verbatim}
x = very &
  long &
  expression
\end{verbatim}
(This is different between
\emph{free format}\index{Fortran!source format!free}
and \emph{fixed format}\index{Fortran!source format!fixed},
where it's the lines after the
first that are marked a continuation, but we don't teach that here.)
\end{itemize}

\begin{slide}{Statements}
  \label{sl:fstatement}
  \begin{itemize}
  \item One line, one statement
\begin{verbatim}
x = 1
y = 2
\end{verbatim}
\item semicolon to separate multiple statements per line
\begin{verbatim}
x = 1; y = 2
\end{verbatim}
\item Continuation of a line
\begin{verbatim}
x = very &
  long &
  expression
\end{verbatim}
  \end{itemize}
\end{slide}

\Level 1 {Comments}

Fortran knows only single-line
\emph{comments}\index{Fortran!comments},
indicated by an exclamation point:
\begin{verbatim}
x = 1 ! set x to one
\end{verbatim}
Everything from the exclamation point onwards is ignored.

Maybe not entirely obvious: you can have a comment after a
continuation character:
\begin{verbatim}
x = f(a) & ! term1 
  + g(b)   ! term2
\end{verbatim}

\begin{slide}{Comments}
  \label{sl:fcomment}
  \begin{itemize}
  \item Ignore to end of line
\begin{verbatim}
x = 1 ! set x to one
\end{verbatim}
\item comment after continuation
\begin{verbatim}
x = f(a) & ! term1 
  + g(b)   ! term2
\end{verbatim}
  \end{itemize}
\end{slide}

\Level 0 {Variables}

Unlike in C++, where you can declare a variable right before you need
it, Fortran wants its variables declared near the top of the program
or subprogram:
\begin{verbatim}
Program YourProgram
  ! variable declaration
  ! executable code
End Program YourProgram
\end{verbatim}
A variable declaration looks like:
\begin{verbatim}
type, attributes :: name1, name2, ....
\end{verbatim}
where
\begin{itemize}
\item \textit{type} is most commonly \n{integer}, \n{real(4)}, \n{real(8)},
  \n{logical}. See below; section~\ref{sec:ftype}.
\item \textit{attributes} can be \n{dimension}, \n{allocatable},
  \n{intent}, \n{parameters} et cetera.
\item \textit{name} is something you come up with. This has to start
  with a letter.
\end{itemize}

\begin{slide}{Variable declarations}
  \label{sl:fvars}
  \begin{itemize}
  \item Variable declarations at the top of the problem
  \item Variables are implicitly defined. Dangerous, so use:
\begin{verbatim}
implicit none
\end{verbatim}
\item declaration
\begin{verbatim}
type, attributes :: name1, name2, ....
\end{verbatim}
where
\begin{itemize}
\item \textit{type} is most commonly \n{integer}, \n{real(4)}, \n{real(8)},
  \n{logical}
\item \textit{attributes} can be \n{dimension}, \n{allocatable},
  \n{intent}, \n{parameters} et cetera.
  \end{itemize}
\end{itemize}
\end{slide}

\Level 1 {Data types}
\label{sec:ftype}

Fortran has a somewhat unusual treatment of data types: if you don't
specify what data type a variable is, Fortran will deduce it from some
default or user rules. This is a very dangerous practice, so we
advocate putting a line
\begin{verbatim}
implicit none
\end{verbatim}
immediately after any program or subprogram header.

\begin{slide}{Implicit typing}
  \label{sl:fimplicit}
  Fortran does not need variable declarations:\\
  type are determined by name.\\
  This is very dangerous. Use:
\begin{verbatim}
implicit none
\end{verbatim}
  in every program unit.
\end{slide}

You can query how many bytes a data type takes with
\indextermttdef{kind}.

You can set this in the declaration:
%
\verbatimsnippet{fstorage}

\begin{slide}{Floating point types}
  \label{sl:ffloat}
  Indicate number of bytes:
  \verbatimsnippet{fstorage}
\end{slide}

\begin{block}{Numerical precision}
  \label{sl:fprecision48}
  Number of bytes determines numerical precision:
  \begin{itemize}
  \item Computations in 4-byte have relative error $\approx 10^{-6}$
  \item Computations in 8-byte have relative error $\approx 10^{-15}$
  \end{itemize}
  Also different exponent range: max $10^{50}$ and $10^{300}$ respectively.
\end{block}

\begin{block}{Storage size}
  F08: \indextermtt{storage_size} reports number of bits.

  F95: \indextermtt{bit_size} works on integers only.

  \indextermtt{c_sizeof} reports number of bytes, requires
  \indextermtt{iso_c_binding} module.
\end{block}

You can set the precision of floating point numbers with
\indextermttdef{selected_real_kind}, where the argument is the number
of significant digits.

\Level 1 {Constants}

Since there are 4-byte and 8-byte reals, how is that for real
constants? Writing \n{3.14} will usually be a single precision
real. The question is then, if you write
\begin{block}{Single precision constants}
  \label{sl:fsingle}
\begin{verbatim}
real(8) :: x
x = 3.14
y = 6.022e-23
\end{verbatim}
\end{block}

how is that converted to double? This can introduce random junk bits.

Force a constant to be \n{real(8)}:
\begin{block}{Double precision constants}
  \label{sl:fdouble}
  \begin{itemize}
  \item Use a compiler flag such as \n{-r8} to force all reals to be 8-byte.
  \item Write \n{3.14d0}
  \item \verb+x = real(3.14, kind=8)+
  \end{itemize}
\end{block}

\Level 1 {Initialization}

Variables can be initialized in their declaration:
\begin{verbatim}
integer :: i=2
real(4) :: x = 1.5
\end{verbatim}

That this is done at compile time, leading to a common error:
\begin{verbatim}
subroutine foo()
  integer :: i=2
  print *,i
  i = 3
end subroutine foo
\end{verbatim}
On the first subroutine call \n{i} is printed with its initialized
value, but on the second call this initialization is not repeated, and
the previous value of \n{3} is remembered.

\Level 0 {Input/Output, or I/O as we say}
\label{sec:fio}

\begin{block}{I/O routines}
  \label{sl:frw}
  \begin{itemize}
  \item Input: 
\begin{verbatim}
READ *,n
\end{verbatim}
\item Output:
\begin{verbatim}
PRINT *,n
\end{verbatim}
  \end{itemize}
  There is also \n{WRITE}.

  Other syntax for read/write with files and formats.
\end{block}

\Level 0 {Expressions}
\label{sec:fexpr}

\begin{block}{Arithmetic expressions}
  \label{sl:farith}
  \begin{itemize}
  \item Pretty much as in C++
  \item Exception: \n{r**2} for power.
  \item Modulus is a function: \n{MOD(7,3)}.
  \end{itemize}
\end{block}

\begin{block}{Boolean expressions}
  \label{sl:fbool}
  \begin{itemize}
  \item 
    Long form
    \n{.and. .not. .or.}
    \n{.lt. .eq. .ge.}
    \n{.true. .false.}
  \item Short form:
    \verb+< <= == /= > >=+
  \end{itemize}
\end{block}

\begin{block}{Conversion and casting}
  Conversion is done through functions.
  \begin{itemize}
  \item \n{INT}: truncation; \n{NINT} rounding
  \item \n{REAL}, \n{FLOAT}, \n{SNGL}, \n{DBLE}
  \item \n{CMPLX}, \n{CONJG}, \n{AIMIG}
  \end{itemize}
\url{http://userweb.eng.gla.ac.uk/peter.smart/com/com/f77-conv.htm}
\end{block}


\begin{block}{Complex}
  Complex numbers exist
\end{block}

\Level 0 {Review questions}

\begin{exercise}
  \label{ex:f-elt-rev1}
  In declarations
\begin{verbatim}
real(4) :: x
real(8) :: y
\end{verbatim}
what do the \n{4} and \n{8} stand for?

What is the practical implication of using the one or the other?
\end{exercise}

\begin{exercise}
  \label{ex:f-elt-rev2}
  In the following code, if \n{value} is nonzero, what do expect about
  the output?
  %
  \verbatimsnippet{d0}
\end{exercise}
