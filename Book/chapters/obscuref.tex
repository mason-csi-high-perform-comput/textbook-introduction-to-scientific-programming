% -*- latex -*-
%%%%%%%%%%%%%%%%%%%%%%%%%%%%%%%%%%%%%%%%%%%%%%%%%%%%%%%%%%%%%%%%
%%%%
%%%% This TeX file is part of the course
%%%% Introduction to Scientific Programming in C++/Fortran2003
%%%% copyright 2017 Victor Eijkhout eijkhout@tacc.utexas.edu
%%%%
%%%% obscure.tex : other stuff
%%%%
%%%%%%%%%%%%%%%%%%%%%%%%%%%%%%%%%%%%%%%%%%%%%%%%%%%%%%%%%%%%%%%%

\Level 0 {Timing}

Timing is done with the \indextermfort{system_clock} routine.
\begin{itemize}
\item This call gives an integer, counting clock ticks.
\item To convert to seconds, it can also tell you how many ticks per
  second it has: its \indextermbus{timer}{resolution}.
\end{itemize}

\begin{verbatim}
  integer :: clockrate,clock_start,clock_end
  call system_clock(count_rate=clockrate)
  print *,"Ticks per second:",clockrate

  call system_clock(clock_start)
  ! code
  call system_clock(clock_end)
  print *,"Time:",(clock_end-clock_start)/REAL(clockrate)
\end{verbatim}
