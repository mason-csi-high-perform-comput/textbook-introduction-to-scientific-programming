% -*- latex -*-
%%%%%%%%%%%%%%%%%%%%%%%%%%%%%%%%%%%%%%%%%%%%%%%%%%%%%%%%%%%%%%%%
%%%%
%%%% This TeX file is part of the course
%%%% Introduction to Scientific Programming in C++/Fortran2003
%%%% copyright 2017 Victor Eijkhout eijkhout@tacc.utexas.edu
%%%%
%%%% simplex.tex : simple exercises
%%%%
%%%%%%%%%%%%%%%%%%%%%%%%%%%%%%%%%%%%%%%%%%%%%%%%%%%%%%%%%%%%%%%%

\Level 0 {Arithmetic}

\begin{enumerate}
\item
  Given
\begin{verbatim}
int n;
\end{verbatim}
write code that
uses elementary mathematical operators to compute n-cubed: $n^3$.

Do you get the correct result for all~$n$? Explain.
\item What is the output of:
\begin{verbatim}
int m=32, n=17;
cout << n%m << endl;
\end{verbatim}
\end{enumerate}

\Level 0 {Scope}

\begin{enumerate}
\item Is this a valid program?
\begin{verbatim}
void f() { i = 1; }
int main() {
  int i=2;
  f();
  return 0;
}
\end{verbatim}
If yes, what does it do; if no, why not?
\item What is the output of:
\begin{verbatim}
#include <iostream>
using namespace std;
int main() {
  int i=5;
  if (true) { i = 6; }
  cout << i << endl;
  return 0;
}
\end{verbatim}
\item What is the output of:
\begin{verbatim}
#include <iostream>
using namespace std;
int main() {
  int i=5;
  if (true) { int i = 6; }
  cout << i << endl;
  return 0;
}
\end{verbatim}
\item What is the output of:
\begin{verbatim}
#include <iostream>
using namespace std;
int main() {
  int i=2;
  i += /* 5;
  i += */ 6;
  cout << i << endl;
  return 0;
}
\end{verbatim}
\end{enumerate}

\Level 0 {Looping}

\begin{enumerate}
\item Suppose a function
\begin{verbatim}
bool f(int);
\end{verbatim}
is given, which is true for some positive input value. Write a main program that
finds the smallest positive input value for which \n{f} is true.
\item Suppose a function
\begin{verbatim}
bool f(int);
\end{verbatim}
is given, which is true for some negative input value. Write a main program that
finds the (negative) input with smallest absolute value for which \n{f} is true.
\end{enumerate}

\Level 0 {Subprograms}

\begin{exercise}
  \label{ex:flooppos}
  Write the missing function \n{pos_input} that
  \begin{itemize}
  \item reads a number from the user
  \item returns it
  \item and returns whether the number is positive
  \end{itemize}
  in such a way to make this code work:
  
  \snippetwithoutput{flooppos}{funcf}{looppos}

  Hint: is \n{pos_input} a \n{SUBROUTINE} or \n{FUNCTION}? If the
  latter, what is the type of the function result? How many parameters
  does it have otherwise? Where does the variable \n{user_input} get
  its value? So what is the type of the parameter(s) of the function?
\end{exercise}

\Level 0 {Object oriented exercises}

\begin{exercise}
  Why is it a good idea to use an accessor function for the data
  members of a class, rather than
  declaring data members \n{public} and accessing them directly?
\end{exercise}

\begin{exercise}
  You are programming a video game. There are moving elements, and you
  want to have an object for each. Moving elements need to have a
  method \n{move} with an argument that indicates a time duration, and
  this method updates the position of the element, using the speed of
  that object and the duration.

  Supply the missing bits of code.
\begin{verbatim}
class position {
  /* ... */
public:
  position() {};
  position(int initial) { /* ... */ };
  void move(int distance) { /* ... */ };
};
class actor {
protected:
  int speed;
  position current;

public:
  actor() { current = position(0); };
  void move(int duration) {
    /* THIS IS THE EXERCISE: */
    /* write the body of this function */
  };
};
class human : public actor {
public:
  human() // EXERCISE: write the constructor
};
class airplane : public actor {
public:
  airplane() // EXERCISE: write the constructor
};

int main() {
  human Alice;
  airplane Seven47;
  Alice.move( 5 );
  Seven47.move( 5 );
\end{verbatim}
\end{exercise}

\begin{exercise}
  Let a \n{Point} class be given:
\begin{verbatim}
class Point {
private: 
  double x,y;
public:
  Point( double px,double py ) { x = px; y = py; };
  // maybe some more methods
}
\end{verbatim}
How would you design a \n{Set} class so that you could write
\begin{verbatim}
Point p1,p2,p3;
Set pointset;
pointset.add(p1); pointset.add(p2);
\end{verbatim}
%  \verbatimsnippet{setminusset}
\end{exercise}

