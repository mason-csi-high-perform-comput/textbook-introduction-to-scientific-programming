% -*- latex -*-
%%%%%%%%%%%%%%%%%%%%%%%%%%%%%%%%%%%%%%%%%%%%%%%%%%%%%%%%%%%%%%%%
%%%%
%%%% This TeX file is part of the course
%%%% Introduction to Scientific Programming in C++/Fortran2003
%%%% copyright 2017 Victor Eijkhout eijkhout@tacc.utexas.edu
%%%%
%%%% teach.tex : a teachers guide
%%%%
%%%%%%%%%%%%%%%%%%%%%%%%%%%%%%%%%%%%%%%%%%%%%%%%%%%%%%%%%%%%%%%%

This book was written for a one-semester introductory programming course at The
University of Texas at Austin, aimed primarily at students in the
physical and engineering sciences. Thus, examples and exercises are as
much as possible scientifically motivated. This target audience also
explains the inclusion of Fortran.

This book is not encyclopedic. Rather than teaching each topic in its
full glory,
the author has taken a `good practices' approach, where students learn
enough of each topic to become a competent programmer. This serves to
keep this book at a manageable length, and to minimize class lecture
time, emphasizing lab exercises instead.

Even then, there is more material here than can be covered and
practiced in one semester. If only C++ is taught, it is probably
possible to cover the whole of Part~\ref{part:cpp}; for the case where
both C++ and Fortran are taught, we have a suggested timeline below.

\Level 0 {Justification}

The chapters of Part~\ref{part:cpp} and Part~\ref{part:f} are
presented in suggested teaching order. 
Here we briefly justify our (non-standard) sequencing of topics and
outline a timetable for material to be covered in one semester.
Most notably, Object-Oriented programming is covered
before arrays and pointers come very late, if at all.

There are several thoughts behind this. For one, dynamic arrays in~C++
are most easily realized through the \n{std::vector} mechanism, which
requires an understanding of classes. The same goes for
\n{std::string}.

Secondly, in the traditional approach, OOP is taught late, if at all, in
the course. We consider OOP to be an important notion in program
design, and central to~C++, rather than an embellishment on the
traditional C~mechanisms.

\Level 0 {Timeline for a C++/F03 course}

As remarked above, this book is based on a course that teaches both
C++ and Fortran2003. Here we give the timeline used, including some of
the assigned exercises.

For a one semester course of slightly over three months, two months
would be spent on~C++ (see table~\ref{tab:c++plan}), after which a
month is enough to explain Fortran. Remaining time will go to exams
and elective topics.

\begin{table}[ht]
  \begin{tabular}{|l|l|p{1in}|p{1in}p{1in}p{1in}|}
    \hline
    lesson$\#$&date&Topic&Exercises&&\\
    &&&prime&geom&infect\\
    \hline
    1& 1/18 & Statements and expressions&\ref{ex:prime:sum}, \ref{ex:prime:modvar}&&\\
    2& 1/24 & Conditionals&\ref{ex:prime:divtest}&&\\
    3& 1/26 & Control structures&&&\\
    4& 1/31 & Looping&\ref{ex:prime:test}, \ref{ex:prime:test2}, \ref{ex:prime:while}&&\\
    5 && continue&&&\\
    6& 2/06 & Functions&\ref{ex:prime:func}&&\\
    7 && continue&&&\\
    8& 2/12 & I/O (lecture~8)&&&\\
    9& 2/19 & Structs&\ref{ex:prime:struct}&&\\
    10& 2/23 & Objects&\ref{ex:prime:sequence}, \ref{ex:prime:goldbach-pqr}&
        \ref{ex:geom:point}&\\
    11 && continue&&&\\
    12& 2/28 & has-a relation&&\ref{ex:geom:line}, \ref{ex:geom:line2},
        \ref{ex:geom:rect}, \ref{ex:geom:rect2}&\\
    13& 3/02 & inheritance&&\ref{ex:geom:square}, \ref{ex:geom:line3}&\\
    14& 3/07 & Arrays&&&\ref{ex:infect:basic}, \ref{ex:infect:1},
        \ref{ex:infect:2}, \ref{ex:infect:3}\\
    15 && continue&&&\\
    16& 3/23 & Strings&&&\\
    \hline
  \end{tabular}
  \caption{Two-month lesson plan for C++}
  \label{tab:c++plan}
\end{table}

\Level 1 {Project-based teaching}

To an extent it is inevitable that students will do a number of
exercises that are not connected to any earlier or later ones.
However, to give some continuity, we have given some programming
projects that students gradually build towards.

\begin{description}
\item[prime] Prime number testing, culminating in prime number
  sequence objects, and testing a corollary of the Goldbach
  conjecture.
\item[geom] Geometry related concepts; this is mostly an exercise in
  object-oriented programming.
\item[infect] The spreading of infectuous diseases; these are
  exercises in basic array programming.
\end{description}

Rather than including the project exercises in the didactic sections,
each section of these projects list the prerequisite basic sections.

\iffalse
\begin{itemize}
\item[1: 1/18] Statements and expressions
\item[2: 1/24] Conditionals
\item[3: 1/26] Control structures
\item[4: 1/31] Looping
\item[5] continue
\item[6: 2/06] Functions
\item[7] continue
\item[8: 2/12] I/O (lecture~8)
\item[9: 2/19] Structs
\item[10: 2/23] Objects
\item[11] continue
\item[12: 2/28] has-a relation
\item[13: 3/02] inheritance
\item[14: 3/07] Arrays
\item[15] continue
\item[16: 3/23] Strings
\end{itemize}
\fi

\Level 1 {Fortran or advanced topics}

After two months of grounding in OOP programming in~C++, the Fortran
lectures and exercises reprise this sequence, letting the students do
the same exercises in Fortran that they did in~C++.  However, array
mechanisms in Fortran warrant a separate lecture.

If the course focuses solely on~C++, the third month can be devoted to
\begin{itemize}
\item templates,
\item exceptions,
\item namespaces,
\item multiple inheritance,
\item the cpp preprocessor,
\item closures.
\end{itemize}

