% -*- latex -*-
%%%%%%%%%%%%%%%%%%%%%%%%%%%%%%%%%%%%%%%%%%%%%%%%%%%%%%%%%%%%%%%%
%%%%
%%%% This TeX file is part of the course
%%%% Introduction to Scientific Programming in C++/Fortran2003
%%%% copyright 2017 Victor Eijkhout eijkhout@tacc.utexas.edu
%%%%
%%%% warmup.tex : chapter of preliminary stuff
%%%%
%%%%%%%%%%%%%%%%%%%%%%%%%%%%%%%%%%%%%%%%%%%%%%%%%%%%%%%%%%%%%%%%

\Level 0 {Programming environment}

Programming can be done in any number of ways. It is possible to use an
\ac{IDE} such as \indexterm{Xcode} or \indexterm{Visual Studio}, but
for if you're going to be doing some computational science
you should really learn a \indexterm{Unix} variant.
\begin{itemize}
\item If you have a \indexterm{Linux} computer, you are all set.
\item If you have an \indexterm{Apple} computer, it is easy to get you
  going. Install \indextermtt{XQuartz} and a \indexterm{package
    manager} such as \indexterm{homebrew} or \indexterm{macports}.
\item \indextermbus{Microsoft}{Windows} users can use
  \indexterm{putty} but it is probably a better solution to install a
  virtual environment such as \indexterm{VMware}
  (\url{http://www.vmware.com/}) or
  \indexterm{Virtualbox} (\url{https://www.virtualbox.org/}).
\end{itemize}

Next, you should know a text editor. The two most common ones are
\indexterm{vi} and \indexterm{emacs}.

\Level 1 {Language support in your editor}
\label{sec:editor-mode}

The author of this book is very much in favour of the
\indextermdef{emacs} editor. The main reason is its support for
programming languages. Most of the time it will detect what language a
file is written in, based on the file extension:
\begin{itemize}
\item \n{cxx,cpp,cc} for C++, and
\item \n{f90,F90} for Fortran.
\end{itemize}
If your editor somehow doesn't detect the language, you can add a line
at the top of the file:
\begin{verbatim}
// -*- c++ -*-
\end{verbatim}
for C++ mode, and 
\begin{verbatim}
! -*- f90 -*-
\end{verbatim}
for Fortran mode.

Main advantages are automatic indentation (C++ and Fortran) and
supplying block end statements (Fortran). The editor will also apply
`syntax colouring' to indicate the difference between keywords and variables.

\Level 0 {Compiling}

The word `program' is ambiguous. Part of the time it means the
\indexterm{source code}: the text that you type, using a text
editor. And part of the time it means the \indexterm{executable}, a
totally unreadable version of your source code, that can be understood and
executed by the computer. The process of turning your source code into
an executable is called \indexterm{compiling}, and it requires
something called a \indexterm{compiler}. (So who wrote the source code
for the compiler? Good question.)

Here is the workflow for program development
\begin{enumerate}
\item You think about how to solve your program
\item You write code using an editor. That gives you a source file.
\item You compile your code. That gives you an executable.

  Oh, make that: you try to compile,
  because there will probably be compiler errors: places where you
  sin against the language syntax.
\item You run your code. Chances are it will not do exactly what you
  intended, so you go back to the editing step.
\end{enumerate}

